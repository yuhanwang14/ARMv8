\documentclass[11pt]{article}

\usepackage{fullpage}
\usepackage[outputdir=out]{minted}
\usemintedstyle{friendly}

\begin{document}

\title{ARM Interim Checkpoint Report}
\author{Bingshi Xu, Henry Wu, Lei Ye, Yuhan Wang}

\maketitle

% On or before Friday 7 June 2024 at 19:00, you should submit on Scientia under COMP40009 a PDF not more than one sheet of paper, i.e. 2x A4 pages maximum, containing a summary that outlines your group working and implementation of the Emulator (Part I) only. This report must be in L A TEX (this is assessed) and include:
% • A statement on how you have split the work between group members and how you are coordinating your work.
% • A discussion on how well you think the group is working and how you imagine it might need to change for the later tasks.
% • How you have structured your emulator, and what bits you think you will be able to reuse for the assembler.
% • A discussion on implementation tasks that you think you will find difficult / challenging later on, and how you are working to mitigate these.

\section{Group Organisation}

The work has been divided into two main parts. Yuhan Wang (yw8123) and Lei Ye (ky723) focused on the first part,
which involves the emulator.
Henry Wu (hw2823) and Bingshi Xu (bx223) concentrated on the second part,
which pertains to the assembler.

For the emulator part, Yuhan Wang primarily handled the implementation of single data transfer and load from literal instructions;
Lei Ye primarily dealt with data processing and branch instructions.

\section{Group progress}
We have successfully completed the first two parts, the emulator and assembler, and passed all the tests in the test suite. 
Currently, we are focusing on the third part, which involves the Raspberry Pi, 
as well as the code abstraction and commentation of the previous parts. 
After completing the report, we plan to move on to the extension part.

\section{Implementation Strategies}

\subsection{The program main loop - emulate.c}
This is the entry point of the program, we strived to keep it clean and simple. 
We start by processing the arguments and opening the input and output files for
reading. Then we have our main \texttt{fetch}, \texttt{decode}, 
\texttt{execute} loop. Finally we clean everything up, i.e: \texttt{fclose} and
\texttt{free} everything.

\subsection{Fetching the next instruction}

To represent the register machine, we defined a \texttt{Register} structure in
\texttt{register.h}, which consists of 31 generic registers \texttt{g\_reg},
the zero register, \texttt{PSTATE} as a separate struct, the \texttt{PC}, and
finally the main memory \texttt{ram}.

\section{Notable difficulties}
Since we have fixed all the bugs and problems in the emulator parts, we mainly
talk about two biggest bugs in our program. 

\subsection{Sign extension}
We will decode the bitcode to extract the signed integer, we forgot to perform 
sign extension to the 

\subsection{Refactor word-indexed base to byte-indexed base}
When debugging the general test cases for load and store operations, 
we discovered an issue with the bitwise shift of all the target registers (\texttt{rt}). 
This issue arose because \texttt{ram} and the program counter (\texttt{PC}) are indexed by words (4 bytes), 
rather than by single bytes. 
Consequently, we had to refactor the code from a word-indexed base to a byte-indexed base.

\end{document}
