\documentclass[11pt]{article}

\usepackage{fullpage}
\usepackage[outputdir=out]{minted}
\usemintedstyle{arduino}

\begin{document}

\title{ARM Interim Checkpoint Report}
\author{Bingshi Xu, Henry Wu, Lei Ye, Yuhan Wang}

\maketitle

\section{Group Organisation}

Yuhan Wang (yw8123) and Lei Ye (ky723) focused on Part I, which involves the
emulator. Henry Wu (hw2823) and Bingshi Xu (bx223) concentrated on Part II,
which pertains to the assembler.

For Part I, Yuhan Wang primarily handled the implementation of single data
transfer and load from literal instructions. Lei Ye primarily dealt with data
processing and branch instructions.

\section{Group progress}
We have successfully completed the first two parts, the emulator and assembler,
and passed all the tests in the test suite. Currently, we are working on Part
III, as well as the polishing and documenting code in Part I and II.

\section{Implementation Strategies}

\subsection{Overview}
Our implementation of the emulator is divided in to the following modules:
\begin{itemize}
	\item{emulate:} Contains the main program loop
	\item{fetch:} Fetches the next binary instruction for execution
	\item{decode:} Decodes binary instruction into internal types defined in \texttt{instr.h}
	\item{execute:} Modifies the state of the emulator according to instruction
	\item{instr:} The instruction model
	\item{register:} The register machine model
	\item{utils:} Misc utilities
\end{itemize}

\subsection{The program main loop}
We start by processing the arguments and opening the input and output files.
Then we have our main loop, which repeatedly does \texttt{fetch},
\texttt{decode} and \texttt{execute} until either we are in an invalid state or
we met \texttt{HALT}. Finally we clean everything up, i.e: \texttt{fclose} and
\texttt{free} everything.

\subsection{Representation of the state machine and fetching instructions}

The register machine is represented by the \texttt{Register} structure in
\texttt{register.h}, which consists of 31 generic registers \texttt{g\_reg},
the zero register, \texttt{PSTATE} as a separate struct, the \texttt{PC}, and
finally the main memory \texttt{ram}.

In each cycle, we simply read 32 bits from \texttt{PC} offset from the start of
the ram. The increment of \texttt{PC} is done in \texttt{execute.c}.

\subsection{Decoding instructions into internal representation}

We represent all instructions with a single type \texttt{Instr}. This is
possible through defining Haskell-like GADTs. For example:

\begin{minted}[fontsize=\footnotesize]{C}
typedef struct {
    InstrType type;
    union {
        DpImmed dp_immed;
        DpRegister dp_reg;
        SdTrans sing_data_transfer;
        LoadLiteral load_literal;
        Branch branch;
    };
} Instr;
\end{minted}

Each category is divided further. This model provides clean separation between
different instructions, allowing cleaner code.

To work with the binary instructions, we defined two helper methods,
\texttt{bit\_slice} and \texttt{nth\_bit\_set}. Then we simply slice the
operands out of the instruction and put them into corresponding structs.

\subsection{Executing the instructions}
Thanks to clean separation between modules, \texttt{emulate.c} only needs to
modify the state of the \texttt{Register} and does not need to mangle with
decoding at all. Two nice abstractions we made for this module are
\texttt{R64}/\texttt{R32} and \texttt{RAM\_64}/\texttt{RAM\_64}, which
abstracts the register and main memory.

\section{Notable difficulties}

\subsection{Sign extension}
During the decode of the signed integer operand, we forgot to perform sign
extension to the these signed integers. Which caused undefined behaviour in
many instructions. It's now fixed and abstracted to \texttt{utils.c}.

\subsection{Refactor word-indexed memory to byte-indexed memory}
When debugging the general test cases for load and store operations, we
discovered an issue with the bitwise shift of all the target registers
(\texttt{rt}). This issue arose because our implementation of \texttt{ram} and
the program counter (\texttt{PC}) are indexed by words (4 bytes), rather than
by single bytes. This resulted in a rather difficult late refactor of the
memory model.

\end{document}
