\documentclass[11pt]{article}

\usepackage{fullpage}
\usepackage[outputdir=out]{minted}
\usemintedstyle{friendly}

\begin{document}

\title{ARM Interim Checkpoint Report}
\author{Bingshi Xu, Henry Wu, Lei Ye, Yuhan Wang}

\maketitle

% On or before Friday 7 June 2024 at 19:00, you should submit on Scientia under COMP40009 a PDF not more than one sheet of paper, i.e. 2x A4 pages maximum, containing a summary that outlines your group working and implementation of the Emulator (Part I) only. This report must be in L A TEX (this is assessed) and include:
% • A statement on how you have split the work between group members and how you are coordinating your work.
% • A discussion on how well you think the group is working and how you imagine it might need to change for the later tasks.
% • How you have structured your emulator, and what bits you think you will be able to reuse for the assembler.
% • A discussion on implementation tasks that you think you will find difficult / challenging later on, and how you are working to mitigate these.

\section{Group Organisation}

The work has been divided into two main parts. Yuhan Wang (yw8123) and Lei Ye 
(ky723) focused on the first part, which involves the emulator.
Henry Wu (hw2823) and Bingshi Xu (bx223) concentrated on the second part,
which pertains to the assembler.

For the emulator part, Yuhan Wang primarily handled the implementation of
single data transfer and load from literal instructions;
Lei Ye primarily dealt with data processing and branch instructions.

\section{Group progress}
We have successfully completed the first two parts, the emulator and assembler,
and passed all the tests in the test suite. 
Currently, we are focusing on the third part, which involves the Raspberry Pi, 
as well as the code abstraction and commentation of the previous parts. 
After completing the report, we plan to move on to the extension part.

\section{Implementation Strategies}

\subsection{The program main loop - emulate.c}
This is the entry point of the program, we strived to keep it clean and simple. 
We start by processing the arguments and opening the input and output files for
reading. Then we have our main \texttt{fetch}, \texttt{decode}, 
\texttt{execute} loop. Finally we clean everything up, i.e: \texttt{fclose} and
\texttt{free} everything.

\subsection{Representation of the state machine and fetching instructions}

In our program, the arm cpu is modeled as a register machine, we defined a
\texttt{Register} structure in \texttt{register.h}, which consists of 31
generic registers \texttt{g\_reg}, the zero register, \texttt{PSTATE} as a
separate struct, the \texttt{PC}, and finally the main memory \texttt{ram}.

In each cycle, we simply read 32 bits from \texttt{PC} offset from the start of
the ram.

\subsection{Decoding instructions into internal representation}

We represent all instructions with a single type \texttt{Instr}. This is 
possible throught defining Haskell-like GADTs consisting a \texttt{XType} enum
representing the subtype, and a union of different cases. We divide
\texttt{Instr} via its operand layout, and further divide those by similar 
means. This resulted in clean separation between each case and very fine typing
to work with, especially in the emulate module.

To work with the binary instructions, we defined two helper methods,
\texttt{bit\_slice} and \texttt{nth\_bit\_set}. Then we simply slice the
operands out of the instruction and put them into corresponding structs.

\subsection{Executing the instructions}
Because we separated the cases nicely, \texttt{emulate.c} is entirely
distangled from the further decoding, allowing for cleaner code and
easier debugging. Two nice abstractions we made for this module are 
\texttt{R64}/\texttt{R32} and \texttt{RAM\_64}/\texttt{RAM\_64}, which 
abstracts the register and main memory so we don't have enigmatic dereferencing
and casting everywhere.

\section{Notable difficulties}
Since we have fixed all the bugs and problems in the emulator parts, we mainly
talk about two biggest bugs in our program. 

\subsection{Sign extension}
We will decode the bitcode to extract the signed integers, we forgot to perform 
sign extension to the these signed integers. Then we got a wrng answer when we need these integer to 
be our offset in either branch instructions or pre/post index type and load from literal instructions. 
Therefore, we choose to abstract sign extension feature to \texttt{utils.c} and perform it 
to all the signed integers. 

\subsection{Refactor word-indexed base to byte-indexed base}
When debugging the general test cases for load and store operations, 
we discovered an issue with the bitwise shift of all the target registers (\texttt{rt}). 
This issue arose because \texttt{ram} and the program counter (\texttt{PC}) are indexed by words (4 bytes), 
rather than by single bytes. 
Consequently, we had to refactor the code from a word-indexed base to a byte-indexed base.

\end{document}
