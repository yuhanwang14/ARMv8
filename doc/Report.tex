\documentclass[11pt]{article}

\usepackage{fullpage}
\usepackage[outputdir=out]{minted}
\usemintedstyle{arduino}

\usepackage{fullpage}

\begin{document}
% REPORT
% You should write a short 3-page (i.e. maximum 6x A4 sides) LATEX final report documenting your project
% (and its Extension, for Computing groups), which must contain:
% • How you have structured and implemented your Assembler. (Please note that the Emulator has
% already been described in the Interim Checkpoint document and should not be re-added here).
% • How you have implemented Part III, i.e. making the green LED blink, on the provided Raspberry Pi.
% • A description of how you have tested your implementation, and a discussion of how effective you
% believe this testing to be.
% • A group reflection on programming in a group. This should include a discussion on how effective
% you believe your way of communicating and splitting work between the group was, and things you
% would do differently or keep the same next time.
% • Individual reflections (at least one paragraph per group member). Using your first Peer Assessment
% feedback, and other experiences, reflect on how you feel you fitted into the group. For example, what
% your strengths and weaknesses turned out to be compared to what you thought they might be or
% things you would do differently or maintain when working with a different group of people.

% PRESENTATION
% Along with your report, you should also prepare a pre-recorded 15 minute video of your group present-
% ing your project and extension. Sticking to the 15 minute duration is an assessed aspect of your presenta-
% tion. The presentation slides that you used for your talk should be submitted to Scientia in PDF format. You
% should also submit a URL.txt file with the address of your video.
% You will be uploading your presentation video on Panopto, under the Assignment folder of COMP40009, here: 
% https://imperial.cloud.panopto.eu/Panopto/Pages/Sessions/List.aspx#folderID=4925854c-5a7f-4266-ad79-b17800ad103c,
% adding “Group 34" in the title. On Week 9, you will have your final assessed meeting with your Mentor, which will in-
% volve a brief Q&A session on your submitted presentation and project.
% Your presentation should discuss
% your entire project and extension, and all group members must participate and present. The presentation
% should include:
% • A demonstration of the provided test-suite running against your emulator and assembler. You may
% use this as an opportunity to briefly discuss any interesting or failing test cases if appropriate (approx.
% 12 minutes).
% • A reflection on the assignment. For example, how well do you think it went, how well did you work
% as a group. Are there any particular experiences or insights that this exercise has given you into
% programming with your peers? What would you do differently next time, or seek to maintain for
% future group programming assignments? (approx. 3 minutes)

\title{ARM Final Report}
\author{Bingshi Xu(bx223), Henry Wu(hw2823), Lei Ye(ky723), Yuhan Wang(yw8123)}

\maketitle

\section{Assembler Implementation Strategies}
\subsection{Overview}
\begin{itemize}
	\item \texttt{assemble}: Contains the main program loop
	\item \texttt{two pass}: Implementation of the two-pass method
	\item \texttt{parser}: resolves alias, Parses single instruction, Calls helper parsers
	\item \texttt{branch parser}: Parses branching instructions
	\item \texttt{dir parser}: Parses directives ('.int' only for this implementation)
	\item \texttt{dp parser}: Parses data processing instructions
	\item \texttt{dt parser}: Parses data transfer instructions
	\item util: Misc utilities
\end{itemize}

\subsection{Two-pass}
Two-pass is effectively the lexing part of the assembler. The module
substitutes all labels into number literals. To achieve this, we traverse the
file first, trimming empty lines and spaces, and also records label and their
corresponding number in an array based map. This is the \texttt{first pass}. In
the second pass, we take note of instructions that could contain a label and
substitute them. This is the \texttt{second pass}. After the lexing, the parser
module receives in memory strings directly, which helps code separation.

\subsection{Parsers}
The parser accepts one line of instruction each time, and dispatches the
appropriate particular parser for each case. Due to the special argument format
of the address for Data transfer instructions, a special handler will handle
separation of arguments for DTIs and determining the addressing mode for such
instructions. For all other types, the main parser separates the arguments
using the \texttt{strtok} function from \texttt{<string.h>}, then check if any
alias presents in this line, changing the opcode and the arguments accordingly.
The standard parsers for each type of instructions is then called to parse the
binary code for the instruction.

\section{Testing Our Implementation}

We used the given test cases and passed all test cases. We also used valgrind
on \texttt{all} test cases to detect memory leaks, no leak is detected in the
most recent version.
% TODO(bx223): maybe talk about debug prints? This section is a bit too short

\section{Part III: Blinking the LED}
\paragraph{Controlling the hardware}
The task was achieved by using pins on the Pi that can be controlled via
software. In particular, such a pin could be in one of two states: high voltage
and low voltage. Which would make the LED "on" and "off" respectively. All of
the pin controls involve setting particular bits in the memory. A particularity
is that it only works if you do a 32 bit set, not a 64 bit one. This was not
mentioned in the spec and it took trial and error to find out.
\paragraph{Setting a time interval} Because we did not implement all of the
instruction set, we could not \texttt{sleep} like in higher level languages.
Instead, we just have a loop doing addition:

\begin{minted}{C}
wait1:
add x0, x0, #1
cmp x0, x2
b.ne wait1
\end{minted}

\section{Group Reflection}

TODO: all\\

\section{Individual Reflections}

\subsection{Bingshi Xu}
Overall, I would consider the collaborating experience demanding but enjoyable.
I've been assigned to work on parser for assembler in our first meet up.
Starting with absolutely no knowledge in C nor assembly language, my first days
on this project had been struggling. In particular, I found understanding the
spec challenging, as many of the details for encoding had been skipped in the
spec for part two, probably because of the information had been covered in part
one. Luckily, my teammates who worked on emulators are helpful and provided
supports on the aspects that I didn't know without working on the emulator.
Moreover, I have learned to use Git when working with others, including how to
use branches and version to avoid unexpected problems followed a bug fix. I've
enjoyed the experience of working in a group and is more confident in doing so
after this project.

\subsection{Henry Wu}
The C group project was challenging yet fun. My main task is to implement the
two-pass methods and the assembler main functions for part II. Initially, my C
programming skills were still developing, and concepts like pointers, memory
leaks, and Git commands were confusing. Besides, I had difficulties and
confusion of the spec content about assembler, and need to continuously scroll
back to previous parts of the spec for more details. However my supportive
group mates came in and helped me grasp these concepts quickly, and through
active learning, I steadily improved my understanding. This not only allowed me
to contribute more effectively to the project but also solidified the value of
continuous learning and teamwork in overcoming programming hurdles.

\subsection{Lei Ye}

This is the first time I worked on a project with others. I learned many things
that I couldn't have learned had I code alone: using git properly, coordinating
progress, assigning issues, reviewing merge requests... I had a lot of fun with
both the project and collaborating with other group members. Although the
implementation of this project was not overwhelmingly complex, it still
deepened my understanding about how executables work, and even did a bit of
hardware in Part III! It's a great learning project that I wouldn't have done
without this opportunity. The team was wonderful and I definitely look forward
to collaborating in later year projects.

\subsection{Yuhan Wang}
Collaborating with my group members on our project was a highly rewarding
experience. I primarily focused on developing the emulator for Part 1 and
writing the Checkpoint report. At first, I found it challenging to get
comfortable with Git operations and version control, but mastering these skills
proved crucial for effective teamwork. I particularly enjoyed solving technical
issues, such as sign extension and converting word-indexed memory to
byte-indexed memory. Overcoming these challenges gave me a great sense of
satisfaction. Overall, programming as part of a team was both valuable and
enriching for my personal and professional growth.

\end{document}
