\documentclass[11pt]{article}

\usepackage{fullpage}
\usepackage[outputdir=out]{minted}
\usemintedstyle{arduino}

\usepackage{fullpage}

\begin{document}

\title{ARM Final Report}
\author{Bingshi Xu, Henry Wu, Lei Ye, Yuhan Wang}

\maketitle

\section{Assembler Implementation Strategies}

Lorem ipsum dolor sit amet, consectetur adipisicing elit, sed do eiusmod tempor
incididunt ut labore et dolore magna aliqua. Ut enim ad minim veniam, quis
nostrud exercitation ullamco laboris nisi ut aliquip ex ea commodo consequat.
Duis aute irure dolor in reprehenderit in voluptate velit esse cillum dolore eu
fugiat nulla pariatur. Excepteur sint occaecat cupidatat non proident, sunt in
culpa qui officia deserunt mollit anim id est laborum.

\section{Testing Our Implementation}

Yuhan Wang (yw8123) and Lei Ye (ky723) focused on Part I, which involves the emulator. Henry Wu (hw2823) and Bingshi Xu (bx223) concentrated on Part II, which pertains to the assembler.
For Part I, Yuhan Wang primarily handled the implementation of single data transfer and load from literal instructions. Lei Ye primarily dealt with data processing and branch instructions.
For Part II, Henry Wu focused on implementing the two-pass 

\section{Group Reflection}

Yuhan Wang (yw8123) and Lei Ye (ky723) focused on Part I, which involves the emulator. Henry Wu (hw2823) and Bingshi Xu (bx223) concentrated on Part II, which pertains to the assembler.
For Part I, Yuhan Wang primarily handled the implementation of single data transfer and load from literal instructions. Lei Ye primarily dealt with data processing and branch instructions.
For Part II, Henry Wu focused on implementing the two-pass 

\section{Individual Reflections}

\subsection{Bingshi Xu}
Overall, I would consider the collaborating experience demanding but enjoyable. I've been assigned to work on parser for assembler in our first meet up.
Starting with absolutely no knowledge in C nor assembly language, my first days on this project had been struggling. 
In particular, I found understanding the spec challenging, as many of the details for encoding had been skipped in the spec for part two, probably because of 
the information had been covered in part one. 
Luckily, my teammates who worked on emulators are helpful and provided supports on the aspects that I didn't know without working on the emulator.
Moreover, I have learned to use Git when working with others, including how to use branchs and version to avoid unexpected problems followed a bug fix.
I've enjoyed the experience of working in a group and is more confident in doing so after this project.  

\subsection{Henry Wu}

My main task is to implement the two-pass methods and assembler main functions in part II. 
Initially, my C programming was still developing, and concepts like pointers, memory leaks, 
and using Git were confusing. However my supportive group mates came in and helped me grasp 
these concepts quickly, and through active learning, I steadily improved my understanding. 
This not only allowed me to contribute more effectively to the project but also 
solidified the value of continuous learning and teamwork in overcoming programming hurdles.

\subsection{Lei Ye}

Yuhan Wang (yw8123) and Lei Ye (ky723) focused on Part I, which involves the emulator. Henry Wu (hw2823) and Bingshi Xu (bx223) concentrated on Part II, which pertains to the assembler.
For Part I, Yuhan Wang primarily handled the implementation of single data transfer and load from literal instructions. Lei Ye primarily dealt with data processing and branch instructions.
For Part II, Henry Wu focused on implementing the two-pass 

\subsection{Yuhan Wang}
Collaborating with my group members on our project was a highly rewarding 
experience. I primarily focused on developing the emulator for Part 1 and 
writing the Checkpoint report. At first, I found it challenging to get 
comfortable with Git operations and version control, but mastering these 
skills proved crucial for effective teamwork. I particularly enjoyed solving 
technical issues, such as sign extension and converting word-indexed memory 
to byte-indexed memory. Overcoming these challenges gave me a great sense of 
satisfaction. Overall, programming as part of a team was both valuable and 
enriching for my personal and professional growth.


\end{document}
