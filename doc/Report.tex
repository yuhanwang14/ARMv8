\documentclass[11pt]{article}

\usepackage{fullpage}
\usepackage[outputdir=out]{minted}
\usemintedstyle{arduino}

\usepackage{fullpage}

\begin{document}

\title{ARM Final Report}
\author{Bingshi Xu, Henry Wu, Lei Ye, Yuhan Wang}

\maketitle

\section{Assembler Implementation Strategies}

\subsection{Overview}
\begin{itemize}
    \item assemble: Contains the main program loop
    \item two pass: Implementation of the two-pass method
    \item parser: resolves alias, Parses single instruction, Calls helper parsers
    \item branch parser: Parses branching instructions
    \item dir parser: Parses directives ('.int' only for this implementation)
    \item dp parser: Parses data processing instructions
    \item dt parser: Parses data transfer instructions
    \item util: Misc utilities
\end{itemize}

\subsection{Two-pass}
The two-pass accepts a file name in string as input and store the path in a local variable. 
Then first-pass is called, which counts the total number of line(excluding empty and label line) 
in the file and records all the labels' information into a table, removes line comments. After that, the input file is split
into separate lines and replaces labels into addresses.

\subsection{parser}
The parser accepts one line of instruction each time, and dispatches the appropriate particular parser for each case.
Due to the special argument format of the address for Data transfer instructions, a special handler will handle seperation of arguemnts for DTIs and determining the 
addressing mode for such instructions. For all other types, the main parser seperates the arguments using the strtok function from <String.h>, then check if any alias presents in 
this line, changing the opcode and the arguments accordingly. The standard parsers for each type of instructions is then called to parse the binary code for the instruction.

\section{Testing Our Implementation}

We used the given test cases and passed all test cases.
We also used valgrind on all test cases to detect memory leaks, no leak is detected in the most recent version.

\section{Difficulties}
\subsection{understanding the spec}
Our group decided to split the team into two, hence the assembler team didn't work on the emulator part, which caused problems due to lack of understanding of how the machine codes work.
Also, part 2 is poorly documented and we had to continuously refer back to the part one contents which created problems.
\subsection{classification of instruction types}
Many of the instructions have similar arguments, and deciding how to seperate these instructions took quite much time.
\subsection{memory leaks}
Due to lack of experience, our first working version leaked enormous memory, and we had to rewrite many modules as our original prototypes contain too many mallocs and are difficult to free.


\section{Group Reflection}

Yuhan Wang (yw8123) and Lei Ye (ky723) focused on Part I, which involves the emulator. Henry Wu (hw2823) and Bingshi Xu (bx223) concentrated on Part II, which pertains to the assembler.
For Part I, Yuhan Wang primarily handled the implementation of single data transfer and load from literal instructions. Lei Ye primarily dealt with data processing and branch instructions.
For Part II, Henry Wu focused on implementing the two-pass 

\section{Individual Reflections}

\subsection{Bingshi Xu}
Overall, I would consider the collaborating experience demanding but enjoyable. I've been assigned to work on parser for assembler in our first meet up.
Starting with absolutely no knowledge in C nor assembly language, my first days on this project had been struggling. 
In particular, I found understanding the spec challenging, as many of the details for encoding had been skipped in the spec for part two, probably because of 
the information had been covered in part one. 
Luckily, my teammates who worked on emulators are helpful and provided supports on the aspects that I didn't know without working on the emulator.
Moreover, I have learned to use Git when working with others, including how to use branchs and version to avoid unexpected problems followed a bug fix.
I've enjoyed the experience of working in a group and is more confident in doing so after this project.  

\subsection{Henry Wu}

My main task is to implement the two-pass methods and assembler main functions in part II. 
Initially, my C programming was still developing, and concepts like pointers, memory leaks, 
and using Git were confusing. However my supportive group mates came in and helped me grasp 
these concepts quickly, and through active learning, I steadily improved my understanding. 
This not only allowed me to contribute more effectively to the project but also 
solidified the value of continuous learning and teamwork in overcoming programming hurdles.

\subsection{Lei Ye}

Yuhan Wang (yw8123) and Lei Ye (ky723) focused on Part I, which involves the emulator. Henry Wu (hw2823) and Bingshi Xu (bx223) concentrated on Part II, which pertains to the assembler.
For Part I, Yuhan Wang primarily handled the implementation of single data transfer and load from literal instructions. Lei Ye primarily dealt with data processing and branch instructions.
For Part II, Henry Wu focused on implementing the two-pass 

\subsection{Yuhan Wang}
Collaborating with my group members on our project was a highly rewarding 
experience. I primarily focused on developing the emulator for Part 1 and 
writing the Checkpoint report. At first, I found it challenging to get 
comfortable with Git operations and version control, but mastering these 
skills proved crucial for effective teamwork. I particularly enjoyed solving 
technical issues, such as sign extension and converting word-indexed memory 
to byte-indexed memory. Overcoming these challenges gave me a great sense of 
satisfaction. Overall, programming as part of a team was both valuable and 
enriching for my personal and professional growth.


\end{document}
